\chapter{Conclusão}\label{cap:conclusao}

	A partir dos resultados experimentais apresentados na seção \xx, é possível realizar uma avaliação comparativa em relação às características propostos na seção \ref{sec:intro:objetivo}. O projeto em módulos torna o escalável, possibilitando a utilização de diversos circuitos de medição em um sistema de controle único. Devido a este fator, o custo final do sistema é dependente do número de módulos utilizado. Apesar disso, a priorização da qualidade das medidas faz com que o projeto não obedeça o que quesito de baixo custo.


	\section{Aprimoramentos do Projeto e Trabalhos Futuros}\label{sec:conclusao:melhoras}

	O programa myGrapher apresentou desempenho satisfatório para uma taxa de amostragem de 1,2 kSPS. Entretanto, são listadas a seguir algumas modificações visando a melhoria do programa, no seu formato visual quanto no seu desempenho de processamento e cálculo:
	\index{\textit{myGrapher}}

		\begin{itemize}
			\item Adição de controles de cor sobre o conteúdo do gráfico;
			\item Adição de controles de escala nos eixos X e Y;
			\item Adição de característica de rolagem de dados no tempo;
			\item Adição de vetor de tempo no armazenamento e gravação dos dados;
			\item Adição de outros formatos de número na exportação de dados;
			\item Modificação da função de processamento gráfico para uma biblioteca mais leve;
			\item Disponibilizar o programa em uma página na internet, possibilitando que o programa realize atualizações de forma automática;
			\item Adição de cálculo de tamanho de exportação do arquivo de dados (utilizar dados das ultimas transferências);
			\item Retirada dos \textit{buffers} intermediários, gravação direta na memória;
			\item Correção do valor RMS calculado;
			\item Bug na gravação de dados em CSV, delimitador duplo ao final de uma página de memória;
			\item Correção do indicador \textit{Skip Counter}.
		\end{itemize}
		\index{CSV}

	A partir da observação do circuito do primeiro protótipo, é possível reduzir o número de isoladores necessários pode ser reduzido pela remoção dos sinais de ganho G0 a G4. Um registrador de deslocamento pode ser utilizado para realizar a conversão de um sinal serial correspondente aos ganhos G0 a G4, reduzindo o número de sinais necessários para 2 (CLK e DTA). Além disso, estes podem ser acoplados aos barramento SPI. O compartilhamento do sinal DTA com o sinal MOSI reduz de 5 para 1 o número de sinais necessários para isolação do controle de ganho do módulo.

	Também, poucos dados de corrente foram adquiridos, e em níveis baixos ao considerar a faixa de operação projetada, conduzindo assim à redução da confiabilidade dos resultados obtidos. Além disso, estes resultados foram comparados com um equipamento laboratorial não certificado como referência de calibração. A realização de mais comparativos com outros medidores e com níveis de corrente mais elevados pode reduzir o efeito destes fatores. Contudo, a calibração com uma referência certificada é o procedimento correto e, portanto, o mais indicado.

	Por fim, em um posterior desenvolvimento de outros protótipos, uma caixa ou invólucro deve ser considerada, afim de possibilitar sua classificação nos graus de segurança (IP) da norma IEC 60529;

	Em futuras versões, uma configuração da taxa de aquisição pode ser implementada como melhoria do sistema \textit{firmware} do sistema de controle. Esta configuração possibilita o melhor aproveitamento de desempenho do conversor ADC utilizado.
	\index{DTA}
	\index{DTR}
	\index{SPI}

