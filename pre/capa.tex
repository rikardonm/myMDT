% faz-se a capa
\begin{capa}
	\centering
	\onehalfspacing
	\fonte{14}
	\MakeUppercase{\bfseries\universidade}\par
	\MakeUppercase{\bfseries\centroCurso}\par
	\MakeUppercase{\bfseries\departamentoPrograma}\par
	\vspace{160pt}
	\fonte{16}
	\textsc{\bfseries\tituloTrabalho}\par
	\vspace{144pt}
	\fonte{14}
	\textsc{\bfseries\grauTrabalho}\par
	\vspace{98pt}
	{\bfseries\autor}\par
	%bottom of page
	\vfill
	{\bfseries \cidadeCEP}\par
	{\bfseries \ano}\par

	%reseta contagem das páginas, just for
	%\setcounter{page}{1}
\end{capa}
% se tem lombada na capa, ela é impressa em cima.
% se não, imprimir em folha separada, ou nem imprimir?
\newlength{\leftlength}
\newlength{\rightlength}
\iflombadaNaCapa
	\setlength{\leftlength}{0cm}
	\setlength{\rightlength}{2cm}
	\colorlet{lombadaQuadro}{white}
	\colorlet{lombadaTexto}{black}
\else
	\clearpage
	\setlength{\leftlength}{2cm}
	\setlength{\rightlength}{5cm}
	\colorlet{lombadaQuadro}{black}
	\colorlet{lombadaTexto}{white}
\fi
	\begin{tikzpicture}[remember picture, overlay]
		\node (left1)	at	(current page.south west) [xshift=\leftlength] {};
		\node (right1)	at	(current page.north west) [xshift=\rightlength] {};
		\draw	[fill=lombadaQuadro, draw=none] (left1)	rectangle	node [rotate=270, text width=\paperheight]
			{\textcolor{lombadaTexto}{\hspace*{3cm}\fonte{20}\departamentoProgramaSigla , \estadoCEP\hfill\autorCitacao\hfill\ano\hspace*{4cm}}}	(right1);
		%\node (lombada) at (current page.center) [rotate=270] {\textcolor{white}{\departamentoProgramaSigla ,\quad\estadoCEP\qquad\autorCitacao\quad\ano}};
	\end{tikzpicture}

